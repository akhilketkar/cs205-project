\documentclass[12pt]{article}
\usepackage{fullpage,amsmath,amsfonts,mathpazo,microtype,nicefrac,graphicx,fancyhdr,listings,hyperref,csvsimple}
\author{Akhil Ketkar Arjun Sanghvi \\
	\texttt{akhilketkar@g.harvard.edu} \texttt{asanghvi@g.harvard.edu}}
\pagestyle{fancy}
\fancyhf{}
\rhead{AM205 2014 Final Project -- AK AS}
\lstset{
	language=Python,
	showstringspaces=false,
	formfeed=\newpage,
	tabsize=4,
	commentstyle=\itshape,
	basicstyle=\ttfamily\scriptsize,
	morekeywords={models, lambda, forms}
}
        
% Macro definitions
\newcommand{\N}{\mathbb{N}}
\newcommand{\Z}{\mathbb{Z}}
\newcommand{\Q}{\mathbb{Q}}
\newcommand{\R}{\mathbb{R}}
\newcommand{\p}{\partial}
\newcommand{\Trans}{\mathsf{T}}

% include graphics
% \includegraphics[width=0.8\textwidth]{figureProb40}

% include code 
% 

% include csv
% \csvautotabular{charge_output.csv}


\title{Spectral Analysis of Enron Email Data}
\begin{document}
\maketitle
\section{Introduction}
	The Enron email dataset is interesting because it contains real email data from employees at a major organization that was involved in a massive fraud. The dataset contains a large amount of information that can be used to answer a number of interesting questions in areas such as Social Network Analysis, Organizational Behavior, such as: who are the key actors in the information network, are there communities in within the network, how do these features of the network evolve over time, does information flow over a network look different in a "crisis" etc. In addition to network or graph theoretic techniques, the dataset can be analyzed from an NLP perspective. 

\section{Brief Background on Enron}

\section{Data and Resulting Graphs}
	What dataset was used \\
	Preprocessing done \\ 
	Kind of graphs produced \\
	Basic metrics on the graph such as degree distributions, diameter, components etc.

\section{Centrality Measures} 
	\subsection{Eigenvalue Based Centrality Measures} Describe how the methods work and why they are a good measure of centrality to begin with
	\begin{enumerate}
		\item Eigenvalue Centrality
		\item Katz Centrality
		\item PageRank 
	\end{enumerate}
	
	\subsection{Other measures of Centrality} Why might they be useful
	\begin{enumerate}
	\item In and Out degree
	\item Betweenness centrality
	\end{enumerate}

\section{Communities} Brief background on homophily and measures of it on a graph. Idea of modularity
	\subsection{Spectral Partitioning} 
		Laplacian of the graph. Why is it important.  \\
		How can we use the second lowest eigenvector of the laplacian to split the partition. \\
		Application to the Enron dataset \\
	\subsection{Modularity Maximization Using Spectral Techniques} 
		Modularity matrix \\
		Maximization Problem and solution using Lagrangian \\
		Communities found in the Enron Dataset 
	\subsection{Singular Value Decomposiiton}
		Low Rank Approximation of Martix \\
		Community Detection using SVD \\

\section{Evolution Across Time} How various  features have evolved over time

\section{Conclusion}

\end{document}
